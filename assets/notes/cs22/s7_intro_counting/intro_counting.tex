\documentclass[12pt,letterpaper]{article}

%Packages
\usepackage{amsmath}
\usepackage{amsthm}
\usepackage{amsfonts}
\usepackage{amssymb}
\usepackage{amscd}
\usepackage{dsfont}
\usepackage{enumerate}
\usepackage{fancyhdr}
\usepackage{mathrsfs}
\usepackage{bbm}
\usepackage{framed}
\usepackage{mdframed}
\usepackage{cancel}
\usepackage{float}
\usepackage{mathtools}
%\usepackage[]{mcode}
\usepackage{graphicx}
\usepackage{tikz}
\usetikzlibrary{automata,arrows}

%Page formatting
\usepackage[letterpaper,voffset=-.5in,bmargin=3cm,footskip=1cm]{geometry}
\setlength{\parindent}{0.0in}
\setlength{\parskip}{0.1in}
\allowdisplaybreaks
\headheight 15pt
\headsep 10pt

% Common Commands
\newcommand\N{\mathbb N}
\newcommand\Z{\mathbb Z}
\newcommand\R{\mathbb R}
\newcommand\Q{\mathbb Q}
\newcommand\lcm{\operatorname{lcm}}
\newcommand\setbuilder[2]{\ensuremath{\left\{#1\;\middle|\;#2\right\}}}
\newcommand\E{\operatorname{E}}
\newcommand\V{\operatorname{V}}
\newcommand\Pow{\ensuremath{\operatorname{\mathcal{P}}}}

\DeclarePairedDelimiter\ceil{\lceil}{\rceil}
\DeclarePairedDelimiter\floor{\lfloor}{\rfloor}

% 22 style
\newcommand\hint[1]{\textbf{Hint}: #1}
\newcommand\note[1]{\textbf{Note}: #1}
\newenvironment{22enumerate}{\begin{enumerate}[a.]\itemsep0em}{\end{enumerate}}
\newenvironment{22itemize}{\begin{itemize}\itemsep0em}{\end{itemize}}
\fancypagestyle{firstpagestyle} {
  \renewcommand{\headrulewidth}{0pt}%
  \lhead{\textbf{CSCI 0220}}%
  \chead{\textbf{Discrete Structures and Probability}}%
  \rhead{Klivans}%
}

\pagestyle{fancyplain}

\newcommand\EX{\mathds{E}}
\newcommand\PR{\mathds{P}}
\newcommand\zlam{Z_\lambda}
\DeclareMathOperator*{\argmin}{arg\,min}

\begin{document}
  \thispagestyle{firstpagestyle}
  \begin{center}
    {\large \textbf{Recitation 7}}
    
    {\large Big O's and Pigeons}

  \end{center}

	\section*{Review}



	\textbf{Defn 1:} (Big O). We say $f(x) \in O(g(x))$ if $g$ grows `faster' than $f$.

		Formally, if $f(x) \in O(g(x))$ then $\exists c,k$ such that $|f(x)| < c|g(x)|$ for all $x > k$. 

	\textbf{Thm:} (Pigeonhole principle). If we take k + 1 pigeons, and put them into k holes, some hole must contain at least two pigeons.

	More generally, if we put $n$ objects into $k$ boxes, then some box has at least $\lceil n/k \rceil$ objects.


	\subsection*{Warm-Up}

	\begin{enumerate}[a.]

		\item Answer true or false for all of the following
		\begin{enumerate}[i.]
			\item The relation $R = \{(f,g) \ | \ f(n) \in O(g(n))\}$ is reflexive.
			\item The relation $R = \{(f,g) \ | \ f(n) \in O(g(n))\}$  is transitive.
			\item The relation $R = \{(f,g) \ | \ f(n) \in O(g(n))\}$  is an equivelance relation.
			\item $2n \in O(n)$
			\item $2n \in O(n^2)$
			\item $n^3 \in O(n^2)$
			\item $100n^2 \in O(n^2)$
			\item $n^{100} \in O(2^n)$
	\end{enumerate}

		\pagebreak

		\item $f: A \rightarrow B$ where $|A| > |B|$ and $A$ and $B$ are finite. Show $f$ is not an injection.
		
		\begin{mdframed}
			\vspace{6cm}
		\end{mdframed}

		\item $f: A \rightarrow B$ where $|A| = |B|$ where $A$ and $B$ are finite. Show that if $f$ is not a surjection then $f$ is not an injection.
		
		\begin{mdframed}
			\vspace{6cm}
		\end{mdframed}

		\textbf{Checkpoint - Call over a TA}

		\pagebreak
	

		\item There are 9 planes and 13 airports. Each day every plane visits 3 different airports. Prove that there must exist one airport each day which is visited by at least 3 planes. 

		\begin{mdframed}
			\vspace{4cm}
		\end{mdframed}	

		\item Given an arbitrary sequence of 100 integers, prove that there exists a consecutive subsequence whose sum is divisible by 100.

		\hint Start by considering consecutive subsequences starting at the first element.

		\begin{mdframed}
			\vspace{12cm}
		\end{mdframed}

		
		
\end{enumerate}	

		\textbf{Checkpoint - Call over a TA}

		\pagebreak

		\section*{Pigeonhole Problems}
	
		\subsection*{Ice Cream Social Problem}
		There are $n$ people at the ice cream social. Throughout the night they have a series of dance partners. 

		\begin{enumerate}[i.]
			\item The minimum number of dance partners someone can have is 0. What is the maximum number of dance partners?
			
			\begin{mdframed}
			\vspace{2cm}
			\end{mdframed}

			\item Prove that 2 people will have the same number of dance partners by the end of the night.

			\begin{mdframed}
			\vspace{10cm}
			\end{mdframed}

		\end{enumerate}

\end{document}