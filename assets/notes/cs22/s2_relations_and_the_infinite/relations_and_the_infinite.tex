\documentclass[12pt,letterpaper]{article}

%Packages
\usepackage{amsmath}
\usepackage{amsthm}
\usepackage{amsfonts}
\usepackage{amssymb}
\usepackage{amscd}
\usepackage{dsfont}
\usepackage{enumerate}
\usepackage{fancyhdr}
\usepackage{mathrsfs}
\usepackage{bbm}
\usepackage{framed}
\usepackage{mdframed}
\usepackage{cancel}
\usepackage{float}
\usepackage{mathtools}
%\usepackage[]{mcode}
\usepackage{graphicx}

%Page formatting
\usepackage[letterpaper,voffset=-.5in,bmargin=3cm,footskip=1cm]{geometry}
\setlength{\parindent}{0.0in}
\setlength{\parskip}{0.1in}
\allowdisplaybreaks
\headheight 15pt
\headsep 10pt

% Common Commands
\newcommand\N{\mathbb N}
\newcommand\Z{\mathbb Z}
\newcommand\R{\mathbb R}
\newcommand\Q{\mathbb Q}
\newcommand\lcm{\operatorname{lcm}}
\newcommand\setbuilder[2]{\ensuremath{\left\{#1\;\middle|\;#2\right\}}}
\newcommand\E{\operatorname{E}}
\newcommand\V{\operatorname{V}}
\newcommand\Pow{\ensuremath{\operatorname{\mathcal{P}}}}

\DeclarePairedDelimiter\ceil{\lceil}{\rceil}
\DeclarePairedDelimiter\floor{\lfloor}{\rfloor}

\newcommand\definition{\textbf{Definition: }}
\newcommand\proposition{\textbf{Proposition: }}

% 22 style
\newcommand\hint[1]{\textbf{Hint}: #1}
\newcommand\note[1]{\textbf{Note}: #1}
\newenvironment{22enumerate}{\begin{enumerate}[a.]\itemsep0em}{\end{enumerate}}
\newenvironment{22itemize}{\begin{itemize}\itemsep0em}{\end{itemize}}
\fancypagestyle{firstpagestyle} {
  \renewcommand{\headrulewidth}{0pt}%
  \lhead{\textbf{CSCI 0220}}%
  \chead{\textbf{Discrete Structures and Probability}}%
  \rhead{Klivans}%
}

\pagestyle{fancyplain}

\newcommand\EX{\mathds{E}}
\newcommand\PR{\mathds{P}}
\newcommand\zlam{Z_\lambda}
\DeclareMathOperator*{\argmin}{arg\,min}


%%%%%%%%%%%%%%%% COMPILE WITH \soltrue or \solfalse %%%%%%%%%%%%%%%%%%%%%%%%%%%%%
\newif\ifsol
\soltrue  % SOLUTIONS
\solfalse % NO SOLUTIONS
%%%%%%%%%%%%%%%%%%%%%%%%%%%%%%%%%%%%%%%%%%%%%%%%%%%%%%%%%%%%%%

%%%%%%%%%%%%%%%% solution help %%%%%%%%%%%%%%%%%%%%%
\newcommand{\solu}[2]{ \begin{mdframed} \ifsol #2 \else \vspace{#1} \fi \end{mdframed} }
\newcommand{\solm}[1]{\ifsol \textit{(#1)} \fi}


\begin{document}
  \thispagestyle{firstpagestyle}
  \begin{center}
    {\large \textbf{Recitation 2}}
    
    {\large Relations, Functions, and the Infinite}
  \end{center}
  
  

      \section*{Part 1: Relations}

	 \subsection*{Operations on Sets}
	
	
\begin{itemize}
    \item $A \cup B = \{ x \ |\  x \in A$ or $ x \in B \}$ (Union)
    \item $A \cap B = \{ x\ |\ x \in A$ and $ x \in B \}$ (Intersection)
    \item $A \setminus B = \{ x\ |\ x \in A$ and $ x \notin B \}$ (Set Difference)
    \item $ \overline{A} = A^c = \{ x\ |\ x \not\in A \}$ (Set Complement)
	\item $A \times B = \{(a,b) \ | \ a \in A $ and $ b \in B\}$ (Cartesian Product)
\end{itemize} 

	\subsection*{Definitions}

\textbf{Defn 1:} A \textit{relation} $R$ on the sets $A$ and $B$ is a subset of the Cartesian product $A \times B$. 

A relation $R$ on the set $A$ is a subset of the Cartesian product $A \times A$. 

Notationally, if an ordered pair $(a,b)$ is in the relation $R$, we can write $(a,b) \in R$ or $aRb$.

\textbf{Defn 2:} An \textit{equivalence relation} is a relation that is reflexive, symmetric, and transitive.

\textbf{Defn 3:} A \textit{partition} of a set $A$ is a collection of subsets $B_1, \ldots, B_k$ of $A$ s.t. every element of $A$ is in some subset $B_i$, but no two subsets share an element.

\textbf{Defn 4:} Let $R$ be an equivalence relation on $A$. Then the \textit{equivalence class} of $a \in A$, denoted $\left[a\right]_R$, is $\{x \mid x\in A, (x,a) \in R\}$.

\proposition The equivalence classes of a relation $R$ on $A$ form a partition of $A$.
	
	\subsection*{Relations Quick Guide and Common Mistakes}
	
	Discuss the following definitions and common mistakes before your proceed.
	
	\begin{description}
		\item[Reflexive]{A relation $R$ on set $A$ is reflexive if $(a,a)\in R$ \textbf{for every} $a \in A$. 

Common mistake: Consider the relation $R$ on the set of students at Brown where two students are related if they took CS15 at the same time. You might think that this relation is reflexive since a student is clearly took CS15 at the same time as themself. However, there is at least one student $s$ who hasn't taken CS15 and therefore $(s,s) \not\in R$. As a result, $R$ is not reflexive. For a relation to be reflexive, $(s,s) \in R$ for every $s$ in the set.}
	
	  \item[Symmetric]{A relation $R$ on $A$ is \textit{symmetric} if $\forall a,b \in A$, if $(a,b) \in R$ then $(b,a) \in R$.

		Common mistake: Consider the relation $R = \{(1,1),(2,2)\}$ on the set $\{1,2\}$. This relation \textbf{is symmetric}. Since $(1,2) \not\in R$, it is not required that $(2,1) \in R$.}

	\item[Transitive]{A relation $R$ on $A$ is \textit{transitive} if $\forall a,b,c \in A$, if $(a,b) \in R$ and $(b,c) \in R$ then $(a,c) \in R$. 

	Common mistake: Consider the relation $R = \{(1,2),(1,1)\}$ on the set $\{1,2\}$. This relation \textbf{is transitive}. Can you see why?}

	\end{description}

	\subsection*{Warm-Up}
	

	\begin{enumerate}[a.]
		\item Consider the set $A = \{1,2\}$.
			\begin{enumerate}[i.]
				\item What is the Cartesian product $A \times A$?

				\solu{3cm}{$\{(1,1), (1,2),(2,1),(2,2)\}$}

				\item Is  $R_1 = \{(1,1),(1,2),(2,2)\}$ a valid relation on $A$?
				
				\solu{1cm}{Yes}

				\item Is $R_1$ reflexive? Why or why not?
				
				\solu{1cm}{Yes}

				\item Is $R_1$ symmetric? Why or why not?


				\solu{1cm}{No, needs $(2,1)$.}

				\item Is  $R_0 = \{\}$ a valid relation on $A$?


				\solu{1cm}{Yes}

				\item Is $R_0$ symmetric? Why or why not?
				
				\solu{1cm}{Yes}

				\item Is $R_0$ transitive? Why or why not?

				\solu{1cm}{Yes}

				\item 	$R_0$ is not an equivalence relation because it is not reflexive. Can you see why?
				
				\solu{1cm}{No. Not reflexive.}
	
			\end{enumerate}

		\section*{Checkpoint - Call a TA over}

		\item Consider the set $B$ of all students at Brown. For each of the following relations on $B$, state if they are reflexive, symmetric, or transitive. If it is an equivelance relation then list the equivalence classes. \textbf{No formal proof needed, just discuss with your group}.

		\begin{enumerate}[i.]
			\item Two students are related if they are the same age (e.g. 21).


				\solu{3cm}{Reflexive, symmetric, and transitive. Therefore equivelance relation. Equivelance classes are students of each age. }


			\item$s_1$ and $s_2$ are students and $(s_1,s_2) \in R$ if $s_1$ is younger than $s_2$.

				\solu{3cm}{Transitive but not reflexive or symmetric.}

		\item Two students are related if they are studying anthropology.
		
				\solu{3cm}{Symmetric and transitive but not reflexive}


	  	\item Two students are related if they go to Brown.


				\solu{3cm}{Reflexive, symmetric, and transitive. Therefore equivelance relation. One equivelance class which consists of all students at Brown.}
	
		\end{enumerate}

	\end{enumerate}

\textbf{Checkpoint - Call a TA over}
\newpage

	\section*{Part 2: Functions and The Infinite}

      \subsection*{Definitions}

      \textbf{Defn 1:} $f : X \rightarrow Y$ is an \textbf{injection} from set $X$ to set $Y$ if for every $y \in Y$, there is \textit{at most one} $x \in X$ such that $f(x) = y$. Equivalently if $x_1 \not= x_2$ then $f(x_1) \not= f(x_2)$.

      An \textbf{injection} is often called \textit{one-to-one} since you are mapping each element in $X$ to a unique element in $Y$. This guarantees that $Y$ must have at least as many elements as $X$, so $|X| \leq |Y|$.

      \textbf{Defn 2:} $f : X \rightarrow Y$ is a \textbf{surjection} from set $X$ to set $Y$ if for every $y \in Y$, there is at least one $x \in X$ such that $f(x) = y$. 

      A \textbf{surjection} is often called \textit{onto} since every single element in $Y$ is mapped to by $f$. This guarantees that $X$ must have at least as many elements as $Y$, so $|X| \geq |Y|$.

      \textbf{Defn 3:} $f : X \rightarrow Y$ is a \textbf{bijection} if it is both an injection and surjection. Since an injection implies $|X| \leq |Y|$ and a surjection implies $|X| \geq |Y|$, a bijection guarantees $|X| = |Y|$.

      \textbf{Defn 4:} $\Pow(S)$ is the set of all subsets of $S$. It is called the \textbf{power set} of $S$. 

      \subsection*{Warm-Up}
 
      For each of the following function, state if $f$ is an injection, surjection, or neither. Also state if it is a bijection. 

      Discuss your solutions.

      \begin{enumerate}[a.]
        \item $f : \{0,1\} \rightarrow \N$

        $f(0) = 1, f(1) = 0$ 

	  \solm{Injective but not surjective.}

        \item $f : \{0,1\} \rightarrow \{0,1\}$

        $f(0) = 1, f(1) = 0$

		\solm{Injective and surjective. Therefore bijective.}

        \item $f : \{0,1\} \rightarrow \{0,1\}$

        $f(0) = 1, f(1) = 1$

		\solm{Not injective or surjective.}

        \item $f : \Z \rightarrow \Z$

        $f(x) = x^2$

		\solm{Not injective or surjective.}		

        \item $f : \text{First Year Students} \rightarrow \text{First Year Dorms}$

        $f(\text{student}) = \text{dorm that student lives in}$

		\solm{Not injective. Surjective.}		

        \item $f : \text{Students} \rightarrow \text{Countries in the World}$

        $f(\text{student}) = \text{country where student is from}$

		\solm{Not injective or surjective.}		

        \item $f : \R \rightarrow \R$

        $f(x) = x$

		\solm{Injective and surjective. Therefore bijective.}		

        \item \textit{Challenge} $f : \R \rightarrow \R$

        $f(x) = \frac{x}{2}$

		\solm{Injective and surjective. Therefore bijective.}		

      \end{enumerate}

	\textbf{Checkpoint - Call a TA over}

      \section*{Section Lesson: Infinite Sizes of Infinity}

      \subsection*{Introduction: Functions as Tables}

      It is sometimes helpful to think of a function as a table where the left column contains all elements in the domain. For example, the function $f: \N \rightarrow \N$ where $f(x) = x^2$ can be represented as follows:


      \begin{center}
      \begin{tabular}{c | c}
        $x$ & $f(x)$ \\
        \hline $0$ & $0$ \\
         $1$ & $1$ \\
         $2$ & $4$ \\
         $3$ & $9$ \\
         $4$ & $16$ \\
         $\vdots$ & $\vdots$ \\
      \end{tabular}
      \end{center}

      We can now redefine injectivity and surjectivity for a function $f: X \rightarrow Y$ as follows:
      \begin{itemize}
        \item $f$ is injective if each element in $Y$ appears in the right column at most once.
        \item $f$ is surjective if all elements of $Y$ appear in the right column at least once.
      \end{itemize}
      
      This gives us better intuition for the important result:

      \textbf{If there is a bijection from $X$ to $Y$ then $|X| = |Y|$}.

      If we have a unique mapping from each element in $X$ to each element in $Y$, and all elements of $Y$ appear in the mapping, it must be the case that $|X| = |Y|$. 

      \subsection*{Extending to the Infinite}

      The same definition applies to infinite sets. If $A$ and $B$ are infinite sets and there exists a bijection $f : A \rightarrow B$ then $A$ and $B$ have the same cardinality.

      Consider the following infinite sets:
      \begin{22itemize}
      \item The natural numbers $\N = \{0, 1, 2, 3, 4, ...\}$
      \item The even natural numbers $E = \{0, 2, 4, 6, 8, ...\}$
      \item The odd natural numbers $O = \{1, 3, 5, 7, 9, ...\}$
      \end{22itemize}

      \textbf{Claim: $|E| = |O|$}. There are as many even numbers as odd numbers.

      \textbf{Proof:} This is intuitive, but we can prove it by giving a bijection $f: E \rightarrow O$.

      \begin{center}
      \begin{tabular}{c | c}
        $x$ & $f(x) = x + 1$ \\
        \hline $0$ & $1$ \\
         $2$ & $3$ \\
         $4$ & $5$ \\
         $6$ & $7$ \\
         $\vdots$ & $\vdots$ \\
      \end{tabular}
      \end{center}\qed

      However, what's more suprising is that $|E| = |\N|$.

      This brings us to our first problem:

      \begin{22enumerate}
      \item Show that there are just as many even numbers as there are natural numbers
        by giving a bijection $f:\N \rightarrow \mathbb{E}$.
        You do not need to prove that this is a bijection.

        \solu{4cm}{$f(n) = 2n$}

      \end{22enumerate}

      \subsection*{\textit{Challenge} - Different Sizes of Infinity}

      We can use a similar method to show that
         there are \textbf{different ``sizes" of infinity}.
       You are going to show this by proving that for any infinite set $S$ the
        following is always true:
      \[ |S| < |\Pow(S)| \]

      \begin{22enumerate}
        \setcounter{enumi}{1}

      \item First, prove that $|S| \leq |\Pow(S)|$ by giving
      an injection $g:S\rightarrow \Pow(S)$.

        \solu{4cm}{$f(S) = \{S\}$}


      \end{22enumerate}

      Now you will show that $ |S| \neq |\Pow(S)| $


      This can be proved by contradiction. Assume, for sake of contradiction,
       that the two sets are of equal cardinality and
       therefore there exists a bijection $f:S\rightarrow \Pow(S)$.

      The table below depicts one such bijection.
      (It is just being used as an example,
       and is not relevant to your answer to this problem.)

      \begin{center}
      \begin{tabular}{c | c}
        $s_i \in S$ & $f(s_i) \in \Pow(S)$ \\
        \hline $s_1$ & $\{s_2, s_3, s_5, ... \}$ \\
         $s_2$ & $\{s_2, s_{8769}, ... \}$ \\
         $s_3$ & $\{s_4, s_9, ... \}$ \\
         $s_4$ & $\{\}$ \\
         $s_5$ & $\{s_5\}$ \\
         $\vdots$ & $\vdots$ \\
      \end{tabular}
      \end{center}
      Now consider the following set:
      \[B = \setbuilder{s_i \in S}{s_i \not\in f(s_i)} \]
      In other words, $B$ is the set of all elements in $S$ that are not a
      member of the set that they are mapped to by $f$.

      In the sample bijection provided above, $B = \{s_1, s_3, s_4, ... \}$
      \begin{22enumerate}
              \setcounter{enumi}{2}
      \item Prove that there does not exist an element $s_i \in S$ such that
       $f(s_i) = B$ and therefore there is no bijection between the two sets.
      Given the previous part, what does this
      say about the cardinalities of $S$ and $\Pow(S)$?

      \hint Assume for sake of contradiction that there exists
      an element $s_i \in S$ such that $f(s_i) = B$. Is $s_i \in B$?

        \solu{4cm}{If $s_i$ were in $B$, then by the definition of $B$, $s_i$ is not mapped to an element that contains it. Since $s_i$ maps to $B$, we have reached a contradiction.

If $s_i$ were not in $B$, then $s_i$ does not map to an element that contains it. Therefore $B$ should contain $s_i$. 
t
 }

      \end{22enumerate}

      If you have shown that $ |S| \neq |\Pow(S)| $ and $|S| \leq |\Pow(S)|$,
      you have now shown that $|S| < |\Pow(S)|$ and therefore there are
      different ``sizes" of infinity.

	\textbf{Checkpoint - Call a TA over}

      \subsection*{Infinite Sizes of Infinity}

      \begin{22enumerate}
        \setcounter{enumi}{3}
      \item Prove that there are infinitely many different ``sizes" of infinity.

	\solu{4cm}{Assume for sake of contradiction that there is a largest size of infinity. Now take the power set}


      \end{22enumerate}

      \subsection*{Extra Challenging Problems}

      \begin{22enumerate}
        \setcounter{enumi}{4}
      \item Let $B$ be the set of infinite binary strings. Prove that $|\N| \not= |B|$. (Hint: you have already done this! Think back to class.)

      \solm{Think about subsets.}
      \item Let $C$ be the set of real numbers between 0 and 1. Prove that $|C| = |B|$.

      \solm{Nothing special about base 10.}
      \item Prove by drawing a picture that $|C| = |\R|$. Conclude that $|\Pow{(\N)}| = |\R|$.

      \solm{A function with asymptotes at 0 and 1.}
      \item Prove that the unit line (all real numbers between 0 and 1) has the same cardinality as the unit square (all coordinates $(a,b)$ where $a$ and $b$ are real numbers between 0 and 1).

      \solm{Alternating numbers.}
      \end{22enumerate}




\end{document}