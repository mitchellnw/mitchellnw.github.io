\documentclass[12pt,letterpaper]{article}

%Packages
\usepackage{amsmath}
\usepackage{amsthm}
\usepackage{amsfonts}
\usepackage{amssymb}
\usepackage{amscd}
\usepackage{dsfont}
\usepackage{enumerate}
\usepackage{fancyhdr}
\usepackage{mathrsfs}
\usepackage{bbm}
\usepackage{framed}
\usepackage{mdframed}
\usepackage{cancel}
\usepackage{float}
\usepackage{mathtools}
%\usepackage[]{mcode}
\usepackage{graphicx}

%Page formatting
\usepackage[letterpaper,voffset=-.5in,bmargin=3cm,footskip=1cm]{geometry}
\setlength{\parindent}{0.0in}
\setlength{\parskip}{0.1in}
\allowdisplaybreaks
\headheight 15pt
\headsep 10pt

% Common Commands
\newcommand\N{\mathbb N}
\newcommand\Z{\mathbb Z}
\newcommand\R{\mathbb R}
\newcommand\Q{\mathbb Q}
\newcommand\lcm{\operatorname{lcm}}
\newcommand\setbuilder[2]{\ensuremath{\left\{#1\;\middle|\;#2\right\}}}
\newcommand\E{\operatorname{E}}
\newcommand\V{\operatorname{V}}
\newcommand\Pow{\ensuremath{\operatorname{\mathcal{P}}}}



\DeclarePairedDelimiter\ceil{\lceil}{\rceil}
\DeclarePairedDelimiter\floor{\lfloor}{\rfloor}

% 22 style
\newcommand\hint[1]{\textbf{Hint}: #1}
\newcommand\note[1]{\textbf{Note}: #1}
\newenvironment{22enumerate}{\begin{enumerate}[a.]\itemsep0em}{\end{enumerate}}
\newenvironment{22itemize}{\begin{itemize}\itemsep0em}{\end{itemize}}
\fancypagestyle{firstpagestyle} {
  \renewcommand{\headrulewidth}{0pt}%
  \lhead{\textbf{CSCI 0220}}%
  \chead{\textbf{Discrete Structures and Probability}}%
  \rhead{Klivans}%
}

\pagestyle{fancyplain}

\newcommand\EX{\mathds{E}}
\newcommand\PR{\mathds{P}}
\newcommand\zlam{Z_\lambda}
\DeclareMathOperator*{\argmin}{arg\,min}


%%%%%%%%%%%%%%%% COMPILE WITH \soltrue or \solfalse %%%%%%%%%%%%%%%%%%%%%%%%%%%%%
\newif\ifsol
%\soltrue  % SOLUTIONS
\solfalse % NO SOLUTIONS
%%%%%%%%%%%%%%%%%%%%%%%%%%%%%%%%%%%%%%%%%%%%%%%%%%%%%%%%%%%%%%


\begin{document}
  \thispagestyle{firstpagestyle}
  \begin{center}
    {\large \textbf{Recitation 1}}
    
    {\large Numbers and Proofs}
  \end{center}
  
  

      \section*{Review}

      \subsection*{Definitions}
      \textbf{Defn 1:} A \textbf{set} is a collection of objects with no repitition.

      \textbf{Defn 2:} $B$ is a \textbf{subset} of $A$ if every element in $B$ is also in $A$. This is written as $B \subseteq A$.

      \textbf{Defn 3:} The \textbf{natural numbers} are the set $\N = \{0,1,2,...\}$. The \textbf{integers} $\Z$ are  $\{..., -2, -1, 0, 1, 2, ...\}$.

      \textbf{Defn 4:} A number $n$ is \textbf{even} if $n = 2k$ for some $k \in \Z$. A number $n$ is \textbf{odd} if $n = 2k + 1$ for some $k \in \Z$.

      \textbf{Defn 5:} A number $n$ is \textbf{rational} if $n = \frac{a}{b}$ for some $a,b \in \Z$. 

	\textbf{Defn 6:} $\Pow{(A)}$, called the power set of $A$, is the set of all subsets of $A$. Alternate notation for the power set of $A$ is $2^A$.

      \subsection*{Warm Up}

      Answer true or false to the following problems. Discuss your solutions.

      \begin{enumerate}[a.]
        \item $A$ is any set. Answer true only if the statement is always true.
          \begin{enumerate}[i.]
            \item $A \subseteq A$ \ifsol  (T)  \fi
            \item $\{\} \subseteq A$\ifsol  (T)  \fi
            \item $\{\} \in A$ \ifsol  (F)  \fi
            \item $B:= \{A\}$. $B$ is a set. The notation ``:=" means ``is defined as". \ifsol  (T)  \fi
            \item $C := \{A, A\}$. $C$ is a set. \ifsol  (F, repition) \fi
		 \end{enumerate}
	  \item $A$ is any set and $\Pow{(A)}$ is the set of all subsets of $A$.
		\begin{enumerate}[i.]
		   \item $A \in \Pow{(A)}$  \ifsol  (T)  \fi
		   \item $A \subseteq \Pow{(A)}$  \ifsol  (F)  \fi
		   \item $\emptyset \in \Pow{(A)}$  \ifsol  (T)  \fi
		   \item $\emptyset \subseteq \Pow{(A)}$  \ifsol  (T)  \fi
		   \item $\{A, \emptyset\} \subseteq \Pow{(A)}$  \ifsol  (T)  \fi
          \end{enumerate}
	  \item If $A = \{1,2,4\}$ then $\{2,4\} \in \Pow{(A)}$  \ifsol  (T)  \fi
        \item $\N \subseteq \Z$  \ifsol  (T)  \fi
        \item $\{0, 1, 9\} \subseteq \N$  \ifsol  (T)  \fi
        \item $\{-1.5, 9\} \subseteq \Z$  \ifsol  (F)  \fi
        \item $S$ is the set of students in CS22. $B$ is the set of students at Brown. Jerry is a student in CS22.
        \begin{enumerate}[i.]
          \item $S \subseteq B$  \ifsol  (T)  \fi
          \item \text{Jerry} $\subseteq S$  \ifsol  (F)  \fi
          \item \text{Jerry} $\in S$  \ifsol  (T)  \fi
          \item $\{\text{Jerry}\} \subseteq B$  \ifsol  (T)  \fi
        \end{enumerate}
        \item Let $\mathbb{Q}$ be the set of rational numbers. 
        \begin{enumerate}[i.]
          \item $\mathbb{Q} \cap \N = \N$  \ifsol  (T)  \fi
          \item $\mathbb{Q} \cup \N = \R$  \ifsol  (F)  \fi
        \end{enumerate}
        \item \textit{Challenge:} If $B \subseteq A$ and $\exists \ x \in A$ such that $x \not\in B$ then $|B| < |A|$.  \ifsol  (T if finite, F if infinite)  \fi
      \end{enumerate}

	\textbf{Checkpoint - Call a TA over.}

      \section*{Section Lesson - Proof Techniques and Examples}

      \subsection*{Direct Proof}

      A direct proof occurs when you start with what you know, follow a series of steps, and end up with what you are trying to prove.

      Here is an example. 

      \textbf{Claim:} If $n$ is odd, then $n^2$ is odd.

      \textbf{Proof (direct):} We know that $n$ is odd, so $n = 2k + 1$ for some $k \in \Z$.

      So $n^2 = (2k+1)(2k+1) = 4k^2 + 4k + 1 = 2(2k^2 + 2k) + 1 = 2m+1$,

      where $m =  2k^2 + 2k$.

      Since $m$ is an integer, $n^2$ is odd. \qed

      \begin{enumerate}[a.]
        \item Prove that the product of an even number and odd number is even. 

        \begin{mdframed}
		\ifsol  
		Consider $n$ odd and $m$ even. $n = 2a+1$ and $m = 2b$.

		Their product is $(2a + 1)(2b) = 4ab + 2b = 2(2ab + b)$.

		Since $2ab + b$ is an integer their product is even.
		\else
        \vspace{4cm}
		\fi
        \end{mdframed}
		

        \item Prove that the product of two rational numbers is rational.
		
		\hint{The product of two integers is an integer.}

        \begin{mdframed}
		\ifsol
		Consider $n = \frac{a}{b}$ and $m = \frac{c}{d}$.

		Their product is $nm = \frac{ac}{bd}$. Since the product of two integers is an integer, we have just expressed $nm$ as the ratio of two integers. Therefore $nm$ is rational.
	 	\else
        \vspace{4cm}
		\fi
        \end{mdframed}

      \end{enumerate}

      \section*{Counterexample}

      Counterexamples help us prove that something is not true.

      For example, suppose Ben makes the claim that if $xy$ is rational then $x$ and $y$ are rational.

      Jerry can disprove his claim by coming up with a counterexample. For example, if $x = \sqrt{2}$ and $y = \sqrt{2}$, then $xy = 2$, which is rational. 

      However, you \textbf{cannot} prove a claim by showing one example of it. Jerry has not proven that $x$ and $y$ are irrational, he has just shown that they are not always rational.

      For example, the claim ``all CS22 students like ice cream" can be disproved by finding a student who does not like ice cream. Finding this counterexample, however, will not prove that no students like ice cream.

      Your turn. Disprove the following statements with a counterexample.
      \begin{enumerate}[a.]
       \setcounter{enumi}{2}

        \item If $xyz$ is rational, then $x$, $y$, and $z$ are rational.

        \begin{mdframed}
		\ifsol
		Consider $x = \sqrt{2}$, $y = \sqrt{3}$, $z = \sqrt{6}$. $xyz = 6$ which is rational.
		\else
        \vspace{4cm}
		\fi
        \end{mdframed}

        \item $\Pow{(A\cup B)} = \Pow{(A)}\cup \Pow{(B)}$

        \begin{mdframed}
		\ifsol
		Consider $A = \{1\}$ and $B = \{2\}$. 

		Then $\{1,2\} \in \Pow{(A\cup B)}$ but $\{1,2\} \not\in \Pow{(A)}\cup \Pow{(B)}$.
		\else
        \vspace{4cm}
		\fi
        \end{mdframed}

        \item \textit{Challenge - Outside the scope of this class:} All true sentences have proofs. 

        \hint{Consider the sentence ``No proof exists for this sentence."}

        \begin{mdframed}
		\ifsol

		\textbf{No proof exists for the sentence:}

		If a proof existed for the sentence ``No proof exists for this sentence" then the sentence would be true. Thereofre, no proof would exist for it. This is a contradiction. 

		\textbf{The sentence is true:}

	If the sentence were false then a proof would exist for the sentence and the sentence would therefore be true. This is also a contradiction. 

		\textbf{Conclusion:}

	We have therefore shown that there are no proofs for the sentence but the sentence is true.
		\else
        \vspace{6cm}
		\fi
        \end{mdframed}

      \end{enumerate}

	\textbf{Checkpoint - Call a TA over.}

	\section*{Set Element Method}

	How do you prove that $A = B$? First show that $A \subseteq B$ and then you show that $B \subseteq A$. If every element in $A$ is also an element in $B$ and every element in $B$ is also an element of $A$, then $A$ must equal $B$. 

	To show that $A \subseteq B$ you consider an arbitrary element in $A$ and show it is also in $B$. 

	Here is an example.

	\textbf{Claim:} $A \cap (B \cup C) = (A \cap B) \cup (A \cap C)$

	\textbf{Proof:} We will first show that $A \cap (B \cup C) \subseteq (A \cap B) \cup (A \cap C)$

	Consider an arbitrary element $x$ which is in the set $A \cap (B \cup C)$.
	\begin{align*}
		&x \in A \cap (B \cup C) \\
		\Rightarrow &x \in A \text{ and } \left( x \in B \text{ or } x \in C \right) \\
		\Rightarrow & \left( x \in A \text{ and } x \in B \right) \text{ or } \left( x \in A \text{ and } x \in C\right)\\
		\Rightarrow & x\in (A \cap B) \cup (A \cap C)
	\end{align*}

	Therefore  $A \cap (B \cup C) \subseteq (A \cap B) \cup (A \cap C)$.

	Now we will show that  $ (A \cap B) \cup (A \cap C) \subseteq A \cap (B \cup C)$. Consider an arbitrary element $x$ in the set $(A \cap B) \cup (A \cap C)$.
	\begin{align*}
		&x \in (A \cap B) \cup (A \cap C) \\
		\Rightarrow &  \left( x \in A \text{ and } x \in B \right) \text{ or } \left( x \in A \text{ and } x \in C\right)\\
		\Rightarrow & x \in A \text{ and } \left( x \in B \text{ or } x \in C \right) \\
\Rightarrow  & x \in A \cap (B \cup C)
	\end{align*}
	Therefore $ (A \cap B) \cup (A \cap C) \subseteq A \cap (B \cup C)$ and by the set element method we have proved our claim.

	  \begin{enumerate}[a.]
         \setcounter{enumi}{5}

		\item Prove  $\Pow{(A\cap B)} = \Pow{(A)}\cap \Pow{(B)}$

		\begin{mdframed}
		\ifsol
		\begin{align*}
		&x \in \Pow{(A\cap B)} \\	
		\iff & x \subseteq A \cap B \\
		\iff & x \subseteq A \text{ and } x \subseteq B \\
		\iff & x \in \Pow{(A)} \text{ and } x \in \Pow{(B)} \\
		\iff & x \in \Pow{(A)}\cap \Pow{(B)}
		\end{align*} \qed
		\else
		\vspace{6cm}
		\fi
		\end{mdframed}
		
	\end{enumerate}

      \section*{Proof by contradiction}

      Say we have some statement $T$ that we are trying to prove. Here is how we prove it by contradiction:
      \begin{enumerate}
        \item Assume $T$ is not true.
        \item If $T$ is not true, we arrive at a contradiction. 
        \item Since $T$ being false leads us to a contradiction, $T$ must be true.
      \end{enumerate}

      Here is an example.

      \textbf{Claim:} $\N$ is an infinite set.

      \textbf{Proof:} Assume for sake of contradiction that there are a finite number of natural numbers. Then there must be a largest natural number. Say this largest number is $m$. 

      However, $m+1$ is still a natural number, and $m+1$ is larger then $m$. 

      This is a contradiction, as $m$ is the largest natural number.

      Assuming $\N$ was finite led to a contradiction, and therefore $\N$ is infinite. \qed

	Often the claim that you are trying to prove will be of the form ``If $p$ then $q$." If this is the case, then you assume that $q$ is not true and show that if $q$ is not true then $p$ is not true.  This is called the contrapositive. 

      \textbf{Claim:} If $n^2$ is even, then $n$ is even.

      \textbf{Proof:} Assume for sake of contradiction that $n^2$ is even but $n$ is odd. Since $n$ is odd, $n = 2k+1$ for some $k \in \Z$.

      So $n^2 = (2k+1)(2k+1) = 4k^2 + 4k + 1 = 2(2k^2 + 2k) + 1 = 2m+1$.

      Where $m =  2k^2 + 2k$.

      Since $m$ is an integer, $n^2$ is odd. This is a contradiction since $n^2$ is even, and therefore if $n^2$ is even then $n$ must also be even. \qed


      Your turn. Prove the following by contradiction:

      \begin{enumerate}[a.]
         \setcounter{enumi}{6}

        \item 2 is an even number. (Use proof by contradiction by assuming 2 is odd.)

        \begin{mdframed}
		\ifsol
			Assume for sake of contradiction that 2 was odd. Then $2 = 2k+1$ for some integer $k$. Then $k = \frac{1}{2}$ which is a contradiction since $k$ is an integer.
		\else
        \vspace{4cm}
		\fi
        \end{mdframed}

        \item \textit{Challenge - Outside the scope of this class:} There is no integer between 0 and 1.
        \hint{Use the fact that every subset of the natural numbers has a smallest element, and that a natural number squared is still a natural number.}

        \begin{mdframed}
		\ifsol
			Consider the set $S$ which is the set of integers between 0 and 1. Assume for sake of contradiction that this set were non-empty. Then $S$ has a smallest element, which we can call $n$. However, $n^2$ is still between 0 and 1 and is still an integer (since the integers are closed under multiplication). Moreover, $n^2$ is smaller than $n$ which is a contradiction as we assumed that $n$ was the smallest integer between 0 and 1. 
		\else
        \vspace{4cm}
		\fi
        \end{mdframed}

        \item \textit{Challenge - Outside the scope of this class:} Consider ``the smallest positive integer not definable in fewer than twelve words". Show that this integer cannot exist.

        \begin{mdframed}
		\ifsol
			Say there was some smallest positive integer not definable in fewer than twelve words. Call this integer $s$. We could then define $s$ in fewer than twelve words by saying ``the smallest positive integer not definable in fewer than twelve words." This is a contradiction.
		\else
        \vspace{4cm}
		\fi
        \end{mdframed}
      \end{enumerate}
  
		\textbf{Checkpoint - Call a TA over.}
\end{document}